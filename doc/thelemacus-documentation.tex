\documentclass[]{book}

% \usepackage{draftwatermark}
\usepackage{authblk}
\usepackage{pdfsync}
\usepackage{boldline}
\usepackage{longtable}
\usepackage[version=3]{acro}
% \usepackage[printonlyused,withpage]{acronym}
\usepackage[numbers,sort&compress]{natbib}
\usepackage{anysize}
\marginsize{2cm}{2cm}{1.cm}{1.cm}
\usepackage{graphicx}
\usepackage{epsfig}
\usepackage{verbatim} 
\usepackage{bm}
\usepackage{etoolbox}
%\usepackage{amssymb}
%\usepackage{amsfonts,amsmath}
\usepackage{multirow}
% \usepackage[T1]{fontenc}
\usepackage{yfonts}
\usepackage{mathrsfs}
\usepackage[colorlinks]{hyperref}
\usepackage{fancyhdr}
\usepackage{relsize}
%  \usepackage{newpxtext,newpxmath}
 \usepackage{booktabs,siunitx}
\usepackage{array,booktabs,ragged2e}
\usepackage{enumitem}   
\usepackage{empheq}
\usepackage[font=small,labelfont=bf]{caption}
\usepackage[title,toc,titletoc,page]{appendix}
\usepackage[capitalise]{cleveref}
\usepackage[most]{tcolorbox}
\usepackage{tocloft}
\usepackage{eurosym}

\acsetup{
  make-links ,
  pages / display = first ,
  pages / fill    = {, }
}


%\SetWatermarkText{CONFIDENTIAL}
%\SetWatermarkScale{.9}
%\SetWatermarkColor{red}%\SetWatermarkColor[rgb]{1,1,1}
%\SetWatermarkAngle{60}

\pagestyle{fancy}
\fancyhf{}
\makeatletter
\fancyhead[L]{Thelemacus documentation}
\makeatother
\fancyhead[R]{\thepage}

% \usepackage{xr}
%\pagestyle{myheadings}
%\usepackage{geometry}
%\externaldocument{sm_2}
%\geometry{tmargin=4cm,bmargin=4cm}

\newcommand{\crefrangeconjunction}{--}
\newcommand{\mycaption}[2]{\caption[#1]{\textbf{#1}. #2}}

\hypersetup{colorlinks=true,
	linkcolor=blue,
	anchorcolor=black,
	citecolor=red,
	urlcolor=black
}

\renewcommand*\oldstylenums[1]{\textosf{#1}}
\addtolength{\cftsecnumwidth}{10pt}


\DeclareAcronym{UTC}{short = UTC, long = Universal Time Coordinated}
\DeclareAcronym{TAI}{short = TAI, long = International Atomic Time}
\DeclareAcronym{nm}{
    short = nm, 
    short-plural-form = {nm},
    long = nautical mile
    }
\DeclareAcronym{km}{
        short = km, 
        short-plural-form = {km},
        long = kilometer
}
\DeclareAcronym{hr}{
        short = h, 
        short-plural-form = {hrs},
        long = hour
}
\DeclareAcronym{min}{
        short = min, 
        short-plural-form = {mins},
        long = minute
}
\DeclareAcronym{sec}{
        short = s, 
        short-plural-form = {s},
        long = second
}
\DeclareAcronym{kt}{
    short = kt, 
    short-plural-form = {kts},
    long = knot
    }
\DeclareAcronym{ft}{
    short = ft, 
    short-plural-form = {ft},
    long = foot,
    long-plural-form = {feet}
    }
\DeclareAcronym{m}{
    short = m, 
    short-plural-form = {m},
    long = meter
    }

\DeclareAcronym{N}{
    short = N, 
    long = North
}
\DeclareAcronym{S}{
    short = N, 
    long = South
}
\DeclareAcronym{E}{
    short = E, 
    long = East
}
\DeclareAcronym{W}{
    short = W, 
    long = West
}
\providecommand{\thel}{Thelemacus\,\,}

% add these two lines to your long preamble    
\DeclareMathAlphabet{\mathcal}{OMS}{cmsy}{m}{n}
\SetMathAlphabet{\mathcal}{bold}{OMS}{cmsy}{b}{n}


\tcbset{highlight math style={boxsep=2mm,colback=white,colframe=black}}


%\newcommand\ddfrac[2]{\frac{\displaystyle #1}{\displaystyle #2}}
%\usepackage{amssymb,graphicx,amsmath}
%\usepackage{times}

% \arraycolsep0.5mm


\newlength{\figsize}
\setlength{\figsize}{1\columnwidth}


\usepackage{xr-hyper} 
% \externaldocument{manuscript}
\usepackage{xcite} 
% \externalcitedocument{manuscript}

\makeatletter 
\arraycolsep0.5mm

%C update title
\title{Thelemacus documentation\\
\large{Version 1.0}}
\author{Michele Castellana}
% \affil[1]{Institut Curie, PSL Research University, CNRS UMR168, France}
% \date{}

% \setlength{\parskip}{5pt plus 0pt minus 0pt}



\begin{document} 

\maketitle 

\pagebreak
\tableofcontents
% \listoffigures

\chapter{Introduction}

\thel  is a navigational-astronomy application, which allows to compute one's position on the surface of the Earth, on both land and sea, based on sextant measurements. \thel is conceived to make the user's life easy, and it is designed for an intuitive and quick use. 

\section{Thelemacus is free software}

\thel is ``free software''; this means that everyone is free to use it. The software is not in the public domain; it is copyrighted and there are conditions on its distribution. These conditions are designed to permit everything that a good cooperating citizen would want to do.


%  \begin{figure}
% \begin{center}
% \includegraphics[scale=1.2]{figures/fig34.eps}
% \mycaption{
% %C Add data for N = 1024 to this figure 
% \label{fig_extr_theta}
% Fit and extrapolation of the stiffness exponent with respect to the number of discretization intervals}{\textbf{A}) 
% Expectation value of the energy gap with respect to the ground state as a function of the number of renormalization-group steps $k$, for a given number of discretization bins $N$,   of samples $S$ and of Robbins-Monro iterations $Q$, and for multiple values of the interaction exponent $\iexp$ (points), in semi-logarithmic scale. The data has been fitted with the function in the right-hand side of \cref{eq180} (lines). 
% \textbf{B}) Stiffness exponent $\theta_N$ (points) obtained as a fit of \textbf{A}. We determined
% $\theta$ (horizontal line) and its error bar (horizontal band) by fitting $\theta_N$ with 
%  \cref{eq_fit_theta} (curves). 
% }
% \end{center}
% \end{figure}

\printacronyms[pages={display=none,seq/use=false}]


%C check overlap between main-text and supplementary-material bibliography
\bibliographystyle{unsrt}
%\addcontentsline{toc}{section}{\refname}
\bibliography{/Users/michelecastellana/Dropbox/my_bibliography/bibliography}



\end{document}